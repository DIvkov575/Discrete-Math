\documentclass[solution,letterpaper]{cs20}
\usepackage{enumerate}
\usepackage{tikz}
\usepackage{pgf}
\usepackage{hyperref}
\usepackage{multicol}
\begin{document}
\header{8}{Asymptotics Counting Counting Subsets}
\begin{problem}
Rank the following functions by their rate of growth, from slowest to fastest.
Specifically, you should rank $f(x)$ after $g(x)$ iff $g(x)=o(f(x))$. Please
indicate the ranking by using the numbers for each function below. If multiple
functions grow asymptotically at the same rate, group their numbers in brackets.
For example, your answer might look like this: 5,2,7,\{3,4\},8,1,9,\{6,10,11\},12.
\begin{multicols}{2}
\begin{enumerate}
\item $10^x$
\item $2^{4x}$
\item $4^{2x}$
\item $e^x$
\item $x^{10}$
\item $x!$
\item $x^x$
\item $5$
\item $\log(x)$
\item $1/x$
\item $x$
\item $\displaystyle \sum^{10}_{n=0} nx^n$
\item $\displaystyle \sum^{10}_{n=0} n^n$
\end{enumerate}
\end{multicols}
\begin{solution}
10, \{13, 8\}, 9, 11, \{5, 12\}, 4, 1, \{2,3\}, 6, 7
\end{solution}
\end{problem}
\newpage

\begin{problem}
Suppose $f,g,h$ are strictly increasing functions from the positive
integers to the positive integers. Prove that if $f(n)$ is $O(g(n))$
then $f(h(n))$ is $O(g(h(n)))$. \\
\begin{solution}
Since \( f(n) \) is \( O(g(n)) \), by definition, there exist constants \( c > 0 \) and \( n_0 \) such that for all \( n \geq n_0 \), \( f(n) \leq c \cdot g(n) \).

We need to show that \( f(h(n)) \) is \( O(g(h(n))) \) i.e. there exist constants \( c' > 0 \) and \( n_1 \) such that for all \( n \geq n_1 \), \( f(h(n)) \leq c' \cdot g(h(n)) \).

Given \( f(n) \leq c \cdot g(n) \) for \( n \geq n_0 \), we substitute \( n \) with \( h(n) \). Since \( h \) is a strictly increasing function from the positive integers to the positive integers, \( h(n) \) will eventually be greater than or equal to \( n_0 \) for sufficiently large \( n \). Let \( n_1 \) be such that \( h(n) \geq n_0 \) for all \( n \geq n_1 \). Such an \( n_1 \) exists because \( h \) is strictly increasing and maps positive integers to positive integers. \\
For all \( n \geq n_1 \), \( h(n) \geq n_0 \). \\
Thus, for \( n \geq n_1 \), \( f(h(n)) \leq c \cdot g(h(n)) \). \\
Let \( c' = c \). Then, \( f(h(n)) \leq c' \cdot g(h(n)) \) for all \( n \geq n_1 \). \\
This shows that \( f(h(n)) \) is \( O(g(h(n))) \) with constants \( c' = c \) and \( n_1 \)\\
\end{solution}
\end{problem}
\newpage


\begin{problem}
Recall that for functions $f$, $g$ on $\mathbb{N}$, $f = O(g)$ iff
\[\exists c \in \mathbb{N}, \exists n_0 \in \mathbb{N}, \forall n \geq n_0\;\;\;c \cdot g(n) \geq |f(n)| \]
For each pair of functions below, determine:
\begin{enumerate}
\item whether $f = O(g)$,
\item whether $g = O(f)$,
\item in cases where a function is $O$ of the other, indicate the \emph{smallest
nonnegative integer} $c$ and for that smallest $c$, the \emph{smallest
corresponding nonnegative integer} $n_0$ ensuring that the condition above is met,
\item in cases where a function is not $O$ of the other, a justification for why
not.
\end{enumerate}
\subproblem $f(n) = n^2$ and $g(n) = 3n$
\subproblem $f(n) = \frac{3n-7}{n + 4}$ and $g(n) = 4$
\subproblem $f(n) = 1 + (n\sin(n\frac{\pi}{2}))^2$ and $g(n) = 3n$ \\
\begin{solution}
(A)  1) False 2) True) 3) $c=1,  n_0 = 2$ \\
(B)  1) True 2) True) 3) $c=2, n_0 = 16$ \\
(C)  1) False 2) False 3) f(n) and g(n) are neither 'O' or '$\Omega$' of each other because of f's oscillating nature. As $n$ increases, $f$ will always oscillate between $n^2$ (larger than $n$) and zero, thereby never reaching a point from where on it will always be less than or greater than $g$. For this same reason, using the provided formula (comparing $\lim_{n \to \infty}{\frac{f(n)}{g(n)}}$for analysis is impossible because the function will never converg as $n$ approaches $\infty$.

\end{solution}
\end{problem}
\newpage

\begin{problem}
    Let $S$ be the set of all 5-digit natural numbers that start with 2 \texttt{1}s or
    end in 2 \texttt{0}s, with no leading zeros. Use the inclusion-exclusion principle
    and the product rule to determine the cardinality of $S$.
    \begin{solution}
        let $S_1$ be the set of 5-digit natural numbers starting with 2 1s \\
        let $S_2$ be the set of 5-digit natural numbers ending in 2 0s with no leading zeros\\
        let $S_3$ be the set of 5-digit natural numbers starting with 2 1s, and ending in two zeros \\
        By the inclusion-exclusion principle $|S| = |S_1| + |S_2| - |S_3|$ \\

        Where $x \in \{1, 0\}$ \\
        let $S_1$ be represented by sequence $\{1,1,x_1, x_2, x_3\}$. Since $x_1, x_2, x_3$ can be 1 of 2 options there are $n^k \text{ or } 2^3$ choices. Therefore, $|S_1| = 8$. \\
        let $S_2$ be represented by sequence $\{1,x_1, x_2, 0, 0\}$. Since $x_1, x_2$ can be 1 of 2 options there are $n^k \text{ or } 2^2$ choices. Therefore, $|S_2| = 4$. \\
        let $S_3$ be represented by sequence $\{1,1, x_2, 0, 0\}$. Since $x_1$ can be 1 of 2 options there are $n^k \text{ or } 2^1$ choices. Therefore, $|S_2| = 2$. \\

        $|S| = 8 + 4 - 2 = 10$ \\
        Therefore, there are 10 strings consisting of 5-digit natural numbers that start with 2 \texttt{1}s or that end in 2 \texttt{0}s, with no leading zeros. \\
    \end{solution}
\end{problem}
\newpage

\begin{problem}
    How many solutions are there to the equation $x_1+x_2+x_3+x_4=10$
    when all of the variables must be non-negative integers.

    \begin{solution}
        To solve the equation \( x_1 + x_2 + x_3 + x_4 = 10 \) where all variables \( x_1, x_2, x_3, \) and \( x_4 \) must be non-negative integers, we can visualize the problem by considering the number 10 as a collection of 10 ones. These ones need to be divided into 4 groups ("bins"), where each group represents one of the variables \( x_1, x_2, x_3, \) and \( x_4 \).
        This can be calculated using combinations: \( \binom{n + k - 1}{k - 1} \), where \( n \) is the total number of ones (10 in this case) and \( k \) is the number of bins (4 in this case). \\
        Substituting the values into the formula, we get: \( \binom{10 + 4 - 1}{4 - 1} = \binom{13}{3} \). \\
        \( \binom{13}{3} = \frac{13!}{3!(13-3)!} = \frac{13!}{3! \cdot 10!}  = 286\) \\
        Therefore, the number of non-negative integer solutions to the equation \( x_1 + x_2 + x_3 + x_4 = 10 \) is 286. \\
    \end{solution}
\end{problem}
\newpage

\begin{problem}
    Consider functions of the form $f : \{1,2,3,4\} \to \{1,2,3,4,5,6\}$.
    \subproblem How many total functions are possible? Explain briefly.
    \subproblem How many of these functions are injective? Explain briefly.
    \subproblem If an injective function is selected at random, what is the probability
    (expressed as a fraction in lowest terms) that the functions is increasing? To be
    increasing means that if $a < b$ then $f(a)< f(b)$ or in other words, the outputs
    get larger as the inputs get larger. Explain briefly.
    \begin{solution}
    (A) \\
    $6^4 = 1296$ because each element from domain can be mapped to one of the 6 co-domain elements. Since elements can be mapped to multiple times, and every elem in domain is mapped, the total becomes ${| \text{co-domain} |}^{| \text{domain} |}$ \\

    (B) \\
    An injective total function is one where every elem in the co-domain is mapped to at most one times and every elem in domain is mapped from.
    The number of injective functions from \( \{1, 2, 3, 4\} \) to \( \{1, 2, 3, 4, 5, 6\} \) is equal to the number of ways of selecting sets of 4 ordered elements from the co-domain. The selected sets represents the co-domain elements being mapped to, while the order is relevant because it makes the mapping domain element significant. \\
    $\frac{6!}{(6-4)!} = 6 * 5 * 4 * 3 = 360$ \\

    (C) \\
    The number of such subsets that map \( \{1, 2, 3, 4\} \) to any subset of \( \{1, 2, 3, 4, 5, 6\} \) of size 4, where the elements are in increasing order is can be found by choosing
    4 elements from \( \{1, 2, 3, 4, 5, 6\} \) and then arrange them in increasing order. Since the problem cares only about a strict ordering and our functions are total and injective, there will only be one order per subset. \\
    The number of unordered subsets (size 4) of 6 elements is $\binom{6}{4} = \frac{6!}{(4!)(2!)} = 6*5/2 = 15$ therefore there are 15 total injective functions which map in an increasing order \\
    The total number of total injective functions is 360 \\
    Therefore the probability of choosing a function with a strictly increasing output ordering, is $\frac{15}{360} = \frac{1}{24}$ \\


    \end{solution}
\end{problem}
\end{document}
