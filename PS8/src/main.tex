\documentclass[solution,letterpaper]{cs20}
\usepackage{enumerate}
\usepackage{tikz}
\usepackage{pgf}
\usepackage{hyperref}
\usepackage{multicol}
\begin{document}
\header{8}{Asymptotics Counting Counting Subsets}
\begin{problem}
Rank the following functions by their rate of growth, from slowest to fastest.
Specifically, you should rank $f(x)$ after $g(x)$ iff $g(x)=o(f(x))$. Please
indicate the ranking by using the numbers for each function below. If multiple
functions grow asymptotically at the same rate, group their numbers in brackets.
For example, your answer might look like this: 5,2,7,\{3,4\},8,1,9,\{6,10,11\},12.
\begin{multicols}{2}
\begin{enumerate}
\item $10^x$
\item $2^{4x}$
\item $4^{2x}$
\item $e^x$
\item $x^{10}$
\item $x!$
\item $x^x$
\item $5$
\item $\log(x)$
\item $1/x$
\item $x$
\item $\displaystyle \sum^{10}_{n=0} nx^n$
\item $\displaystyle \sum^{10}_{n=0} n^n$
\end{enumerate}
\end{multicols}
\begin{solution}
10, \{13, 8\}, 9, 11, \{5, 12\}, 4, 1, \{2,3\}, 6, 7
\end{solution}
\end{problem}
\newpage

\begin{problem}
Suppose $f,g,h$ are strictly increasing functions from the positive
integers to the positive integers. Prove that if $f(n)$ is $O(g(n))$
then $f(h(n))$ is $O(g(h(n)))$. \\
\begin{solution}
Since \( f(n) \) is \( O(g(n)) \), by definition, there exist constants \( c > 0 \) and \( n_0 \) such that for all \( n \geq n_0 \), \( f(n) \leq c \cdot g(n) \).

We need to show that \( f(h(n)) \) is \( O(g(h(n))) \) i.e. there exist constants \( c' > 0 \) and \( n_1 \) such that for all \( n \geq n_1 \), \( f(h(n)) \leq c' \cdot g(h(n)) \).

Given \( f(n) \leq c \cdot g(n) \) for \( n \geq n_0 \), we substitute \( n \) with \( h(n) \). Since \( h \) is a strictly increasing function from the positive integers to the positive integers, \( h(n) \) will eventually be greater than or equal to \( n_0 \) for sufficiently large \( n \). Let \( n_1 \) be such that \( h(n) \geq n_0 \) for all \( n \geq n_1 \). Such an \( n_1 \) exists because \( h \) is strictly increasing and maps positive integers to positive integers. \\
For all \( n \geq n_1 \), \( h(n) \geq n_0 \). \\
Thus, for \( n \geq n_1 \), \( f(h(n)) \leq c \cdot g(h(n)) \). \\
Let \( c' = c \). Then, \( f(h(n)) \leq c' \cdot g(h(n)) \) for all \( n \geq n_1 \). \\
This shows that \( f(h(n)) \) is \( O(g(h(n))) \) with constants \( c' = c \) and \( n_1 \)\\
\end{solution}
\end{problem}
\newpage


\begin{problem}
Recall that for functions $f$, $g$ on $\mathbb{N}$, $f = O(g)$ iff
\[\exists c \in \mathbb{N}, \exists n_0 \in \mathbb{N}, \forall n \geq n_0\;\;\;c \cdot g(n) \geq |f(n)| \]
For each pair of functions below, determine:
\begin{enumerate}
\item whether $f = O(g)$,
\item whether $g = O(f)$,
\item in cases where a function is $O$ of the other, indicate the \emph{smallest
nonnegative integer} $c$ and for that smallest $c$, the \emph{smallest
corresponding nonnegative integer} $n_0$ ensuring that the condition above is met,
\item in cases where a function is not $O$ of the other, a justification for why
not.
\end{enumerate}
\subproblem $f(n) = n^2$ and $g(n) = 3n$
\subproblem $f(n) = \frac{3n-7}{n + 4}$ and $g(n) = 4$
\subproblem $f(n) = 1 + (n\sin(n\frac{\pi}{2}))^2$ and $g(n) = 3n$ \\
\begin{solution}
(A)  1) False 2) True) 3) $c=1,  n_0 = 2$ \\
(B)  1) True 2) True) 3) $c=2, n_0 = 16$ \\
(C)  1) False 2) False 3) f(n) and g(n) are neither 'O' or '$\Omega$' of each other because of f's oscillating nature. As $n$ increases, $f$ will always oscillate between $n^2$ (larger than $n$) and zero, thereby never reaching a point from where on it will always be less than or greater than $g$. For this same reason, using the provided formula (comparing $\lim_{n \to \infty}{\frac{f(n)}{g(n)}}$for analysis is impossible because the function will never converg as $n$ approaches $\infty$.

\end{solution}
\end{problem}
\newpage

\end{document}
