
\documentclass[solution,letterpaper]{cs20}
\usepackage{enumerate}
\usepackage{tikz}
\usepackage{pgf}
\usepackage{hyperref}
\usepackage{amsmath}
\usepackage{qtree}
\usepackage{multicol}

\begin{document}

    \header{7}{Structural Induction  Series Recurrence Relations }


    \begin{problem}

        A set $S$ is defined by the following rules:
        \begin{itemize}
            \item Base case: $(0,1) \in S$.
            \item Constructor Rule: If $(x,y) \in S$, then $(x+1,y+2x+3) \in S$.
        \end{itemize}

        (a) Starting with $(0,1)$, write out the pairs that follow by applying
        the constructor rule five times.

        (b) Based on part (a), give a conjecture about the relationship that holds between the first and second
        coordinates of all pairs in $S$.

        (c) Use structural induction to prove that your conjectured relationship holds for all pairs in $S$.

        \begin{solution}
        (A) \\
        $(0,1) \to (1,4) \to (2, 9) \to (3, 16) \to (4, 25) \to (5, 36)$ \\

        (B) \\
        $\forall (x, y) \in S. x+1)^2 = y$

        (C) \\
        Let us prove that every number in S is in the form $(x, x^2)$ by structural induction.\\
        \textbf{Base Case} (0,1)\\
        \( (x + 1)^2 = y \) \\
        \( (0 + 1)^2 = 1 \) \\
        Base case holds since \( y = 1 \).\\

        \textbf{Inductive Hypothesis}
        Assume the relationship holds for an arbitrary pair \( (x, y) \in S \). That is, assume \( y = (x + 1)^2 \).\\

        \textbf{Inductive Step} \\
        Given our inductive hypothesis, we begin with a starting state $(n, (n+1)^2)$ If we apply the one and only constructor rule we get our inductino state which would be $(n+1, (n+1)^2 + 2n + 3)$. To prove our hypothesis is true we must show that $(n+1)^2 + 2n + 3 = (n+2)^2$. \\
        $(n+2)^2 = (n+1)^2 + 2n + 3$ \\
        $(n+2)^2 = n^2 +2n + 1 + 2n + 3$ \\
        $(n+2)^2 = n^2 +4n +4 $ \\
        $(n+2)^2 = (n+2)^2$  \\
        We have shown that the $(x+1)^2 = y$ holds for all states in states in the structure, by showing it holds true for the one and only constructor case.

        \textbf{Conclusion} \\
        By structural induction, we have shown that for all \( (x, y) \in S \), the relationship \( y = (x + 1)^2 \) holds for all states in the structure.
        \end{solution}
    \end{problem}
    \newpage

    \begin{problem}
        The reflection $Ref(T)$ of a rooted binary tree $T$ (where
        the nodes are labelled) is defined as follows.
        The reflection of a tree $T$ that is a single root node is $T$.
        The reflection of a tree $T = (root,L,R)$ where $L$
        and $R$ are the left and right subtrees of $T$ is
        $Ref(T) = (root,Ref(R),Ref(L))$.

        \subproblem Find the reflection of the tree below.\\

        \includegraphics[width=2.5in]{original_tree}

        \subproblem Prove using structural induction that the reflection of the reflection of a tree is the original tree.

        \begin{solution}
        (A) \\
        \begin{figure}[h]
            \centering
            \includegraphics[width=0.5\linewidth]{IMG_4088 2.JPG}
            \label{fig:enter-label}
        \end{figure}

        (B) \\
        Let us prove that the reflection of the reflection of a tree is the original tree with structural induction. \\

        \textbf{Base Case:} \\
        Consider a tree with a single root node, \( T \). The reflection of \( T \) is itself, i.e., \(\text{Ref}(T) = T\). Applying the reflection operation again, we get \(\text{Ref}(\text{Ref}(T)) = \text{Ref}(T) = T\). Thus, the property holds for a single root node tree. \\

        \textbf{Inductive hypothesis:} \\
        Assume that Ref(Ref($T$)) = $T$ for k height trees \\

        \textbf{Inductive Step:} \\
        Consider a tree \( T \) of height \( k+1 \). \( T \) has a root node, and its left and right subtrees \( L \) and \( R \) are trees of height at most \( k \). We can write \( T \) as \( T = (\text{root}, L, R) \). \\

        \(\text{Ref}(T) = (\text{root}, \text{Ref}(R), \text{Ref}(L))\) \\
        \(\text{Ref}(\text{Ref}(T)) = \text{Ref}(\text{Ref}(\text{root}, \text{Ref}(\text{Ref}(R)), \text{Ref}(\text{Ref}(L)))\) \\

        If we consider each of the k height subtrees $\text{Ref}(\text{Ref}(L))$ and $\text{Ref}(\text{Ref}(R))$, we can notice that these take the form of our hypothesis "Ref(Ref($T$)) = $T$ for k height trees". Therefore $\text{Ref}(\text{Ref}(L)) = L$ and $\text{Ref}(\text{Ref}(R)) = R$. \\

        Therefore \(\text{Ref}(\text{Ref}(T)) = (\text{root}, \text{Ref}(\text{Ref}(L)), \text{Ref}(\text{Ref}(R)))\) is equivalent to \(\text{Ref}(\text{Ref}(T)) = (\text{root}, L, R)\)

        \textbf{Conclusion:}
        Thus, the property holds for trees of height \( k+1 \) if it holds for trees of height \( k \). By structural induction, we have shown that the reflection of the reflection of any rooted binary tree \( T \) is \( T \) itself.
        \end{solution}
    \end{problem}
    \newpage

    \begin{problem}
        We can use perturbation to generate a big cancellation of terms and a closed form for the series $S_n$:

        \begin{align*}
            S_n &= 1 + z + z^2 + \cdots z^n \\
            zS_n &= z + z^2 + z^3 + \cdots + z^{n+1} \\
            S_n - zS_n &= 1 + z^{n+1} \\
            S_n &= \frac{1 - z^{n+1}}{1 - z}
        \end{align*}

        Use this same trick to generate a closed form for the series $T_n$:

        \begin{align*}
            T_n = 1z + 2z^2 + 3z^3 + \cdots + nz^n
        \end{align*}

        \begin{solution}
            $ T_n = 1z + 2z^2 + 3z^3 + \cdots + nz^n$ \\
            $ zT_n = z^2 + 2z^3 + 3z^4 + \cdots + nz^{n+1}$ \\
            $ \frac{1}{z}T_n = 1 + 2z + 3z^2 + \cdots + nz^{n-1}$ \\
            $ \frac{1}{z}T_n - T_n = 1 + z + z^2 + \cdots + z^{n-1} - nz^n$ \\
            $T_n - zT_n = z + z^2 + z^3 + \cdots + z^n - nz^{n+1}$ \\
            $\frac{1}{z}T_n - T_n - T_n - zT_n = 1 - nz^n - z^n + nz^n+1$ \\
            $T_n = \frac{1 - nz^n - z^n + nz^n+1}{\frac{1}{z} - 2 + z}$ \\
        \end{solution}
    \end{problem}
    \newpage

    \begin{problem}
        Find a simple closed form for the series (where here $|p| < 1$)
        $$\sum\limits_{x = 0}^{\infty} \sum\limits_{y = 0}^{\infty} p^{x+y}.$$

        Do the same for
        $$\sum\limits_{x = 0}^{\infty} \sum\limits_{y = 0}^{\infty}
        \sum\limits_{z = 0}^{\infty} p^{x+y+z}.$$

        \begin{solution}

            \(\sum_{x=0}^{\infty} \sum_{y=0}^{\infty} p^{x+y}\). \\


            The inner sum is a geometric series with the first term 1 and the common ratio \( p \), which converges because \(|p| < 1\): \(\sum_{y=0}^{\infty} p^y = \frac{1}{1 - p}\).

            Therefore, the sum becomes: \(\sum_{x=0}^{\infty} p^x \left( \frac{1}{1 - p} \right) = \frac{1}{1 - p} \sum_{x=0}^{\infty} p^x\).

            The outer sum is also a geometric series with the first term 1 and the common ratio \( p \), which converges for \(|p| < 1\): \(\sum_{x=0}^{\infty} p^x = \frac{1}{1 - p}\).
            Therefore, we have: \(\sum_{x=0}^{\infty} \sum_{y=0}^{\infty} p^{x+y} = \frac{1}{1 - p} \cdot \frac{1}{1 - p} = \frac{1}{(1 - p)^2}\). \\

            (A) $\frac{1}{(1-p)^2}$ \\

            \(\sum_{x=0}^{\infty} \sum_{y=0}^{\infty} \sum_{z=0}^{\infty} p^{x+y+z}\). \\

            For fixed \( x \) and \( y \), the innermost sum is: \(\sum_{z=0}^{\infty} p^{x+y+z}\).

            Since \( p^{x+y+z} = p^x \cdot p^y \cdot p^z \), we can factor out \( p^x \cdot p^y \) from the innermost sum: \(\sum_{z=0}^{\infty} p^{x+y+z} = p^x \cdot p^y \sum_{z=0}^{\infty} p^z\).

            The innermost sum is a geometric series with the first term 1 and the common ratio \( p \), which converges for \(|p| < 1\): \(\sum_{z=0}^{\infty} p^z = \frac{1}{1 - p}\).

            Thus, the triple sum becomes: \(\sum_{x=0}^{\infty} \sum_{y=0}^{\infty} p^x \cdot p^y \left( \frac{1}{1 - p} \right) = \frac{1}{1 - p} \sum_{x=0}^{\infty} \sum_{y=0}^{\infty} p^x \cdot p^y\).

            We already know that: \(\sum_{y=0}^{\infty} p^y = \frac{1}{1 - p}\), so the double sum becomes: \(\sum_{x=0}^{\infty} p^x \left( \frac{1}{1 - p} \right) = \frac{1}{1 - p} \sum_{x=0}^{\infty} p^x = \frac{1}{1 - p} \cdot \frac{1}{1 - p}\).

            Therefore, we have: \(\sum_{x=0}^{\infty} \sum_{y=0}^{\infty} \sum_{z=0}^{\infty} p^{x+y+z} = \frac{1}{1 - p} \cdot \frac{1}{1 - p} \cdot \frac{1}{1 - p} = \frac{1}{(1 - p)^3}\).

            (B) $\frac{1}{(1-p)^3}$
        \end{solution}
    \end{problem}
    \newpage

    \begin{problem}

        Rebecca spends \$10 in expenses to start a side hustle that brings in \$40 in the first month. She re-invests 10 dollars of her profit from the previous month into her business on the 1st of each month. Through her hard work, her (gross) earnings each month are double the net profit from the previous month.

        \subproblem Based on the above, Rebecca's earnings in the first and second month are $T(1) = 30$ and $T(2) = 50$, respectively. What are Rebecca's profits in the third and fourth month? Show your work.
        \subproblem Write a recurrence for Rebecca's net earnings. Write both the base case $T(1)$ and the recursive case $T(n)$, where $n$ is the month.
        \subproblem Solve the recurrence. Show your work.


        \begin{solution}
        (A) \\
        $t(3) = 2 * t(2) - 10 = 2 * 50 - 10 = 90$ \\
        $t(4) = 2 * t(3) - 10 = 2 * 90 - 10 = 170$ \\

        (B) \\
        $T(1) = 30; T(n) = 2 \dot T(n-1) - 10$ (for $n \geq 2$) \\

        (C) \\
        Recurrence recurses to a depth of $n$ because w/ each recursion we decrement the argument to T by 1. $n - n + 1 = 1$ therefore we will call n recurrences. \\
        $T(n) = 2^{n}T(n-n+1) + 2^0{10} + 2^1{10} + \cdots + 2^{n}10$\\
        $T(n) = 2^{n}30 + 10\sum{2^i}_{i=0}^{n}$


        \end{solution}
    \end{problem}

\end{document}
